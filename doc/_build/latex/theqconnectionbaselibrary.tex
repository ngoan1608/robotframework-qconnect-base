%% Generated by Sphinx.
\def\sphinxdocclass{report}
\documentclass[letterpaper,10pt,english]{sphinxmanual}
\ifdefined\pdfpxdimen
   \let\sphinxpxdimen\pdfpxdimen\else\newdimen\sphinxpxdimen
\fi \sphinxpxdimen=.75bp\relax
\ifdefined\pdfimageresolution
    \pdfimageresolution= \numexpr \dimexpr1in\relax/\sphinxpxdimen\relax
\fi
%% let collapsible pdf bookmarks panel have high depth per default
\PassOptionsToPackage{bookmarksdepth=5}{hyperref}

\PassOptionsToPackage{warn}{textcomp}
\usepackage[utf8]{inputenc}
\ifdefined\DeclareUnicodeCharacter
% support both utf8 and utf8x syntaxes
  \ifdefined\DeclareUnicodeCharacterAsOptional
    \def\sphinxDUC#1{\DeclareUnicodeCharacter{"#1}}
  \else
    \let\sphinxDUC\DeclareUnicodeCharacter
  \fi
  \sphinxDUC{00A0}{\nobreakspace}
  \sphinxDUC{2500}{\sphinxunichar{2500}}
  \sphinxDUC{2502}{\sphinxunichar{2502}}
  \sphinxDUC{2514}{\sphinxunichar{2514}}
  \sphinxDUC{251C}{\sphinxunichar{251C}}
  \sphinxDUC{2572}{\textbackslash}
\fi
\usepackage{cmap}
\usepackage[T1]{fontenc}
\usepackage{amsmath,amssymb,amstext}
\usepackage{babel}



\usepackage{tgtermes}
\usepackage{tgheros}
\renewcommand{\ttdefault}{txtt}



\usepackage[Bjarne]{fncychap}
\usepackage{sphinx}

\fvset{fontsize=auto}
\usepackage{geometry}


% Include hyperref last.
\usepackage{hyperref}
% Fix anchor placement for figures with captions.
\usepackage{hypcap}% it must be loaded after hyperref.
% Set up styles of URL: it should be placed after hyperref.
\urlstyle{same}

\addto\captionsenglish{\renewcommand{\contentsname}{Contents:}}

\usepackage{sphinxmessages}
\setcounter{tocdepth}{3}
\setcounter{secnumdepth}{3}


\title{The QConnection Base Library}
\date{Feb 07, 2022}
\release{}
\author{Nguyen Huynh Tri Cuong (RBVH/ECM1)}
\newcommand{\sphinxlogo}{\vbox{}}
\renewcommand{\releasename}{}
\makeindex
\begin{document}

\ifdefined\shorthandoff
  \ifnum\catcode`\=\string=\active\shorthandoff{=}\fi
  \ifnum\catcode`\"=\active\shorthandoff{"}\fi
\fi

\pagestyle{empty}
\sphinxmaketitle
\pagestyle{plain}
\sphinxtableofcontents
\pagestyle{normal}
\phantomsection\label{\detokenize{index::doc}}


\sphinxAtStartPar
QConnectBaseLibrary is a connection testing library for \sphinxhref{https://robotframework.org}{Robot
Framework}. Library will be supported to
downloaded from PyPI soon. It provides a mechanism to handle trace log
continously receiving from a connection (such as Raw TCP, SSH, Serial,
etc.) besides sending data back to the other side. It’s especially
efficient for monitoring the overflood response trace log from an
asynchronous trace systems. It is supporting Python 3.7+ and
RobotFramework 3.2+.


\chapter{Table of Contents}
\label{\detokenize{index:table-of-contents}}\begin{itemize}
\item {} 
\sphinxAtStartPar
{\hyperref[\detokenize{index:getting-started}]{\emph{Getting Started}}}

\item {} 
\sphinxAtStartPar
{\hyperref[\detokenize{index:building-and-testing}]{\emph{Usage}}}

\item {} 
\sphinxAtStartPar
{\hyperref[\detokenize{index:example}]{\emph{Example}}}

\item {} 
\sphinxAtStartPar
{\hyperref[\detokenize{index:contribution-guidelines}]{\emph{Contribution Guidelines}}}

\item {} 
\sphinxAtStartPar
{\hyperref[\detokenize{index:configure-Git-and-correct-EOL-handling}]{\emph{Configure Git and correct EOL
handling}}}

\item {} 
\sphinxAtStartPar
{\hyperref[\detokenize{index:documentation}]{\emph{Sourcecode Documentation}}}

\item {} 
\sphinxAtStartPar
{\hyperref[\detokenize{index:feedback}]{\emph{Feedback}}}

\item {} 
\sphinxAtStartPar
{\hyperref[\detokenize{index:about}]{\emph{About}}}
\begin{itemize}
\item {} 
\sphinxAtStartPar
{\hyperref[\detokenize{index:maintainers}]{\emph{Maintainers}}}

\item {} 
\sphinxAtStartPar
{\hyperref[\detokenize{index:contributors}]{\emph{Contributors}}}

\item {} 
\sphinxAtStartPar
{\hyperref[\detokenize{index:3rd-party-licenses}]{\emph{3rd Party Licenses}}}

\item {} 
\sphinxAtStartPar
{\hyperref[\detokenize{index:used-encryption}]{\emph{Used Encryption}}}

\item {} 
\sphinxAtStartPar
{\hyperref[\detokenize{index:license}]{\emph{License}}}

\end{itemize}

\end{itemize}


\chapter{Getting Started}
\label{\detokenize{index:getting-started}}
\sphinxAtStartPar
We have a plan to publish all the sourcecode as OSS in the near future
so that you can downloaded from PyPI. For the current period, you can
checkout all
\sphinxhref{https://sourcecode.socialcoding.bosch.com/projects/ROBFW/repos/robotframework-qconnect-base/browse}{QConnectBaseLibrary}
sourcecode from the Bosch Internal Open Source Repositories.

\sphinxAtStartPar
After checking out the source completely, you can install by running
below command inside \sphinxstylestrong{robotframework\sphinxhyphen{}qconnect\sphinxhyphen{}base} directory.

\begin{sphinxVerbatim}[commandchars=\\\{\}]
\PYG{n}{python} \PYG{n}{setup}\PYG{o}{.}\PYG{n}{py} \PYG{n}{install}
\end{sphinxVerbatim}


\chapter{Usage}
\label{\detokenize{index:usage}}
\sphinxAtStartPar
QConnectBaseLibrary support following keywords for testing connection in RobotFramework.


\section{\sphinxstylestrong{connect}}
\label{\detokenize{index:connect}}\begin{quote}

\sphinxAtStartPar
\sphinxstylestrong{Use for establishing a connection.}

\sphinxAtStartPar
\sphinxstylestrong{Syntax}:
\begin{quote}

\sphinxAtStartPar
\sphinxstylestrong{connect} \sphinxcode{\sphinxupquote{{[}conn\_name{]}   {[}conn\_type{]}   {[}conn\_mode{]}   {[}conn\_conf{]}}}
\sphinxstyleemphasis{(All parameters are required to be in order)}

\sphinxAtStartPar
or

\sphinxAtStartPar
\sphinxstylestrong{connect}
\sphinxcode{\sphinxupquote{conn\_name={[}conn\_name{]}   conn\_type={[}conn\_type{]}   conn\_mode={[}conn\_mode{]}   conn\_conf={[}conn\_conf{]}}}
\sphinxstyleemphasis{(All parameters are assigned by name)}
\end{quote}

\sphinxAtStartPar
\sphinxstylestrong{Arguments}:
\begin{quote}

\sphinxAtStartPar
\sphinxstylestrong{conn\_name}: Name of the connection.

\sphinxAtStartPar
\sphinxstylestrong{conn\_type}: Type of the connection. QConnectBaseLibrary has supported below connection types:
\begin{itemize}
\item {} 
\sphinxAtStartPar
\sphinxstylestrong{TCPIPClient}: Create a Raw TCPIP connection to TCP Server.

\item {} 
\sphinxAtStartPar
\sphinxstylestrong{SSHClient}: Create a client connection to a SSH server.

\item {} 
\sphinxAtStartPar
\sphinxstylestrong{SerialClient}: Create a client connection via Serial Port.

\end{itemize}

\sphinxAtStartPar
\sphinxstylestrong{conn\_mode}: (unused) Mode of a connection type.

\sphinxAtStartPar
\sphinxstylestrong{conn\_conf}: Configurations for making a connection. Depend on \sphinxstylestrong{conn\_type} (Type of Connection), there is a suitable configuration in JSON format as below.
\begin{quote}
\begin{itemize}
\item {} 
\sphinxAtStartPar
\sphinxstylestrong{TCPIPClient}

\end{itemize}

\begin{sphinxVerbatim}[commandchars=\\\{\}]
\PYG{p}{\PYGZob{}}
    \PYG{l+s+s2}{\PYGZdq{}}\PYG{l+s+s2}{address}\PYG{l+s+s2}{\PYGZdq{}}\PYG{p}{:} \PYG{p}{[}\PYG{n}{server} \PYG{n}{host}\PYG{p}{]}\PYG{p}{,} \PYG{c+c1}{\PYGZsh{} Optional. Default value is \PYGZdq{}localhost\PYGZdq{}.}
    \PYG{l+s+s2}{\PYGZdq{}}\PYG{l+s+s2}{port}\PYG{l+s+s2}{\PYGZdq{}}\PYG{p}{:} \PYG{p}{[}\PYG{n}{server} \PYG{n}{port}\PYG{p}{]}     \PYG{c+c1}{\PYGZsh{} Optional. Default value is 1234.}
    \PYG{l+s+s2}{\PYGZdq{}}\PYG{l+s+s2}{logfile}\PYG{l+s+s2}{\PYGZdq{}}\PYG{p}{:} \PYG{p}{[}\PYG{n}{Log} \PYG{n}{file} \PYG{n}{path}\PYG{o}{.} \PYG{n}{Possible} \PYG{n}{values}\PYG{p}{:} \PYG{l+s+s1}{\PYGZsq{}}\PYG{l+s+s1}{nonlog}\PYG{l+s+s1}{\PYGZsq{}}\PYG{p}{,} \PYG{l+s+s1}{\PYGZsq{}}\PYG{l+s+s1}{console}\PYG{l+s+s1}{\PYGZsq{}}\PYG{p}{,} \PYG{o}{\PYGZlt{}}\PYG{n}{user} \PYG{n}{define} \PYG{n}{path}\PYG{o}{\PYGZgt{}}\PYG{p}{]}
 \PYG{p}{\PYGZcb{}}
\end{sphinxVerbatim}
\begin{itemize}
\item {} 
\sphinxAtStartPar
\sphinxstylestrong{SSHClient}

\end{itemize}

\begin{sphinxVerbatim}[commandchars=\\\{\}]
\PYG{p}{\PYGZob{}}
    \PYG{l+s+s2}{\PYGZdq{}}\PYG{l+s+s2}{address}\PYG{l+s+s2}{\PYGZdq{}} \PYG{p}{:} \PYG{p}{[}\PYG{n}{server} \PYG{n}{host}\PYG{p}{]}\PYG{p}{,}  \PYG{c+c1}{\PYGZsh{} Optional. Default value is \PYGZdq{}localhost\PYGZdq{}.}
    \PYG{l+s+s2}{\PYGZdq{}}\PYG{l+s+s2}{port}\PYG{l+s+s2}{\PYGZdq{}} \PYG{p}{:} \PYG{p}{[}\PYG{n}{server} \PYG{n}{host}\PYG{p}{]}\PYG{p}{,}     \PYG{c+c1}{\PYGZsh{} Optional. Default value is 22.}
    \PYG{l+s+s2}{\PYGZdq{}}\PYG{l+s+s2}{username}\PYG{l+s+s2}{\PYGZdq{}} \PYG{p}{:} \PYG{p}{[}\PYG{n}{username}\PYG{p}{]}\PYG{p}{,}    \PYG{c+c1}{\PYGZsh{} Optional. Default value is \PYGZdq{}root\PYGZdq{}.}
    \PYG{l+s+s2}{\PYGZdq{}}\PYG{l+s+s2}{password}\PYG{l+s+s2}{\PYGZdq{}} \PYG{p}{:} \PYG{p}{[}\PYG{n}{password}\PYG{p}{]}\PYG{p}{,}    \PYG{c+c1}{\PYGZsh{} Optional. Default value is \PYGZdq{}\PYGZdq{}.}
    \PYG{l+s+s2}{\PYGZdq{}}\PYG{l+s+s2}{authentication}\PYG{l+s+s2}{\PYGZdq{}} \PYG{p}{:} \PYG{l+s+s2}{\PYGZdq{}}\PYG{l+s+s2}{password}\PYG{l+s+s2}{\PYGZdq{}} \PYG{o}{|} \PYG{l+s+s2}{\PYGZdq{}}\PYG{l+s+s2}{keyfile}\PYG{l+s+s2}{\PYGZdq{}} \PYG{o}{|} \PYG{l+s+s2}{\PYGZdq{}}\PYG{l+s+s2}{passwordkeyfile}\PYG{l+s+s2}{\PYGZdq{}}\PYG{p}{,}  \PYG{c+c1}{\PYGZsh{} Optional. Default value is \PYGZdq{}\PYGZdq{}.}
    \PYG{l+s+s2}{\PYGZdq{}}\PYG{l+s+s2}{key\PYGZus{}filename}\PYG{l+s+s2}{\PYGZdq{}} \PYG{p}{:} \PYG{p}{[}\PYG{n}{filename} \PYG{o+ow}{or} \PYG{n+nb}{list} \PYG{n}{of} \PYG{n}{filenames}\PYG{p}{]}\PYG{p}{,} \PYG{c+c1}{\PYGZsh{} Optional. Default value is null.}
    \PYG{l+s+s2}{\PYGZdq{}}\PYG{l+s+s2}{logfile}\PYG{l+s+s2}{\PYGZdq{}}\PYG{p}{:} \PYG{p}{[}\PYG{n}{Log} \PYG{n}{file} \PYG{n}{path}\PYG{o}{.} \PYG{n}{Possible} \PYG{n}{values}\PYG{p}{:} \PYG{l+s+s1}{\PYGZsq{}}\PYG{l+s+s1}{nonlog}\PYG{l+s+s1}{\PYGZsq{}}\PYG{p}{,} \PYG{l+s+s1}{\PYGZsq{}}\PYG{l+s+s1}{console}\PYG{l+s+s1}{\PYGZsq{}}\PYG{p}{,} \PYG{o}{\PYGZlt{}}\PYG{n}{user} \PYG{n}{define} \PYG{n}{path}\PYG{o}{\PYGZgt{}}\PYG{p}{]}
 \PYG{p}{\PYGZcb{}}
\end{sphinxVerbatim}
\begin{itemize}
\item {} 
\sphinxAtStartPar
\sphinxstylestrong{SerialClient}

\end{itemize}

\begin{sphinxVerbatim}[commandchars=\\\{\}]
\PYG{p}{\PYGZob{}}
    \PYG{l+s+s2}{\PYGZdq{}}\PYG{l+s+s2}{port}\PYG{l+s+s2}{\PYGZdq{}} \PYG{p}{:} \PYG{p}{[}\PYG{n}{comport} \PYG{o+ow}{or} \PYG{n}{null}\PYG{p}{]}\PYG{p}{,}
    \PYG{l+s+s2}{\PYGZdq{}}\PYG{l+s+s2}{baudrate}\PYG{l+s+s2}{\PYGZdq{}} \PYG{p}{:} \PYG{p}{[}\PYG{n}{Baud} \PYG{n}{rate} \PYG{n}{such} \PYG{k}{as} \PYG{l+m+mi}{9600} \PYG{o+ow}{or} \PYG{l+m+mi}{115200} \PYG{n}{etc}\PYG{o}{.}\PYG{p}{]}\PYG{p}{,}
    \PYG{l+s+s2}{\PYGZdq{}}\PYG{l+s+s2}{bytesize}\PYG{l+s+s2}{\PYGZdq{}} \PYG{p}{:} \PYG{p}{[}\PYG{n}{Number} \PYG{n}{of} \PYG{n}{data} \PYG{n}{bits}\PYG{o}{.} \PYG{n}{Possible} \PYG{n}{values}\PYG{p}{:} \PYG{l+m+mi}{5}\PYG{p}{,} \PYG{l+m+mi}{6}\PYG{p}{,} \PYG{l+m+mi}{7}\PYG{p}{,} \PYG{l+m+mi}{8}\PYG{p}{]}\PYG{p}{,}
    \PYG{l+s+s2}{\PYGZdq{}}\PYG{l+s+s2}{stopbits}\PYG{l+s+s2}{\PYGZdq{}} \PYG{p}{:} \PYG{p}{[}\PYG{n}{Number} \PYG{n}{of} \PYG{n}{stop} \PYG{n}{bits}\PYG{o}{.} \PYG{n}{Possible} \PYG{n}{values}\PYG{p}{:} \PYG{l+m+mi}{1}\PYG{p}{,} \PYG{l+m+mf}{1.5}\PYG{p}{,} \PYG{l+m+mi}{2}\PYG{p}{]}\PYG{p}{,}
    \PYG{l+s+s2}{\PYGZdq{}}\PYG{l+s+s2}{parity}\PYG{l+s+s2}{\PYGZdq{}} \PYG{p}{:} \PYG{p}{[}\PYG{n}{Enable} \PYG{n}{parity} \PYG{n}{checking}\PYG{o}{.} \PYG{n}{Possible} \PYG{n}{values}\PYG{p}{:} \PYG{l+s+s1}{\PYGZsq{}}\PYG{l+s+s1}{N}\PYG{l+s+s1}{\PYGZsq{}}\PYG{p}{,} \PYG{l+s+s1}{\PYGZsq{}}\PYG{l+s+s1}{E}\PYG{l+s+s1}{\PYGZsq{}}\PYG{p}{,} \PYG{l+s+s1}{\PYGZsq{}}\PYG{l+s+s1}{O}\PYG{l+s+s1}{\PYGZsq{}}\PYG{p}{,} \PYG{l+s+s1}{\PYGZsq{}}\PYG{l+s+s1}{M}\PYG{l+s+s1}{\PYGZsq{}}\PYG{p}{,} \PYG{l+s+s1}{\PYGZsq{}}\PYG{l+s+s1}{S}\PYG{l+s+s1}{\PYGZsq{}}\PYG{p}{]}\PYG{p}{,}
    \PYG{l+s+s2}{\PYGZdq{}}\PYG{l+s+s2}{rtscts}\PYG{l+s+s2}{\PYGZdq{}} \PYG{p}{:} \PYG{p}{[}\PYG{n}{Enable} \PYG{n}{hardware} \PYG{p}{(}\PYG{n}{RTS}\PYG{o}{/}\PYG{n}{CTS}\PYG{p}{)} \PYG{n}{flow} \PYG{n}{control}\PYG{o}{.}\PYG{p}{]}\PYG{p}{,}
    \PYG{l+s+s2}{\PYGZdq{}}\PYG{l+s+s2}{xonxoff}\PYG{l+s+s2}{\PYGZdq{}} \PYG{p}{:} \PYG{p}{[}\PYG{n}{Enable} \PYG{n}{software} \PYG{n}{flow} \PYG{n}{control}\PYG{o}{.}\PYG{p}{]}\PYG{p}{,}
    \PYG{l+s+s2}{\PYGZdq{}}\PYG{l+s+s2}{logfile}\PYG{l+s+s2}{\PYGZdq{}}\PYG{p}{:} \PYG{p}{[}\PYG{n}{Log} \PYG{n}{file} \PYG{n}{path}\PYG{o}{.} \PYG{n}{Possible} \PYG{n}{values}\PYG{p}{:} \PYG{l+s+s1}{\PYGZsq{}}\PYG{l+s+s1}{nonlog}\PYG{l+s+s1}{\PYGZsq{}}\PYG{p}{,} \PYG{l+s+s1}{\PYGZsq{}}\PYG{l+s+s1}{console}\PYG{l+s+s1}{\PYGZsq{}}\PYG{p}{,} \PYG{o}{\PYGZlt{}}\PYG{n}{user} \PYG{n}{define} \PYG{n}{path}\PYG{o}{\PYGZgt{}}\PYG{p}{]}
 \PYG{p}{\PYGZcb{}}
\end{sphinxVerbatim}
\end{quote}
\end{quote}
\end{quote}


\section{\sphinxstylestrong{disconnect}}
\label{\detokenize{index:disconnect}}\begin{quote}

\sphinxAtStartPar
\sphinxstylestrong{Use for disconnect a connection by name.}

\sphinxAtStartPar
\sphinxstylestrong{Syntax}:
\begin{quote}

\sphinxAtStartPar
\sphinxstylestrong{disconnect} \sphinxcode{\sphinxupquote{conn\_name}}
\end{quote}

\sphinxAtStartPar
\sphinxstylestrong{Arguments}:
\begin{quote}

\sphinxAtStartPar
\sphinxstylestrong{conn\_name}: Name of the connection.
\end{quote}
\end{quote}


\section{\sphinxstylestrong{send command}}
\label{\detokenize{index:send-command}}\begin{quote}

\sphinxAtStartPar
\sphinxstylestrong{Use for sending a command to the other side of connection.}

\sphinxAtStartPar
\sphinxstylestrong{Syntax}:
\begin{quote}

\sphinxAtStartPar
\sphinxstylestrong{send command} \sphinxcode{\sphinxupquote{{[}conn\_name{]}   {[}command{]}}} \sphinxstyleemphasis{(All parameters are
required to be in order)}

\sphinxAtStartPar
or

\sphinxAtStartPar
\sphinxstylestrong{send command}
\sphinxcode{\sphinxupquote{conn\_name={[}conn\_name{]}   command={[}command{]}}} \sphinxstyleemphasis{(All parameters are
assigned by name)} \#\#\#\#\# \sphinxstyleemphasis{Arguments}:
\end{quote}
\end{quote}
\begin{itemize}
\item {} 
\sphinxAtStartPar
\sphinxstylestrong{conn\_name}: Name of the connection.

\item {} 
\sphinxAtStartPar
\sphinxstylestrong{command}: Command to be sent.

\end{itemize}


\section{\sphinxstylestrong{verify}}
\label{\detokenize{index:verify}}\begin{quote}

\sphinxAtStartPar
\sphinxstylestrong{Use for verifying a response from the connection if it matched a pattern.}

\sphinxAtStartPar
\sphinxstylestrong{Syntax}:
\begin{quote}

\sphinxAtStartPar
\sphinxstylestrong{verify}
\sphinxcode{\sphinxupquote{{[}conn\_name{]}   {[}search\_pattern{]}   {[}timeout{]}   {[}fetch\_block{]}  {[}eob\_pattern{]} {[}filter\_pattern{]}  {[}send\_cmd{]}}}\sphinxstyleemphasis{(All
parameters are required to be in order)}

\sphinxAtStartPar
or

\sphinxAtStartPar
\sphinxstylestrong{verify}  \sphinxcode{\sphinxupquote{conn\_name={[}conn\_name{]}   search\_pattern={[}search\_pattern{]}  timeout={[}timeout{]}  fetch\_block={[}fetch\_block{]}  eob\_pattern={[}eob\_pattern{]} filter\_pattern={[}filter\_pattern{]}  send\_cmd={[}send\_cmd{]}}}
\sphinxstyleemphasis{(All parameters are assigned by name)}
\end{quote}

\sphinxAtStartPar
\sphinxstylestrong{Arguments}:
\begin{quote}

\sphinxAtStartPar
\sphinxstylestrong{conn\_name}: Name of the connection.

\sphinxAtStartPar
\sphinxstylestrong{search\_pattern}: Regular expression for matching with the response.

\sphinxAtStartPar
\sphinxstylestrong{timeout}: Timeout for waiting response matching pattern.

\sphinxAtStartPar
\sphinxstylestrong{fetch\_block}: If this value is true, every response line will be put into a block untill a line match \sphinxstylestrong{eob\_pattern} pattern.

\sphinxAtStartPar
\sphinxstylestrong{eob\_pattern}: Regular expression for matching the endline when using \sphinxstylestrong{fetch\_block}.

\sphinxAtStartPar
\sphinxstylestrong{filter\_pattern}: Regular expression for filtering every line of block when using \sphinxstylestrong{fetch\_block}.

\sphinxAtStartPar
\sphinxstylestrong{send\_cmd}: Command to be sent to the other side of connection and waiting for response.
\end{quote}

\sphinxAtStartPar
\sphinxstylestrong{Return value}:
\begin{quote}

\sphinxAtStartPar
\sphinxstylestrong{A corresponding match object if it is found.}

\sphinxAtStartPar
\sphinxstylestrong{E.g.}

\begin{sphinxVerbatim}[commandchars=\\\{\}]
\PYGZdl{}\PYGZob{}result\PYGZcb{} = verify  conn\PYGZus{}name=SSH\PYGZus{}Connection
                     search\PYGZus{}pattern=(?\PYGZlt{}=\PYGZbs{}s).*([0\PYGZhy{}9]..).*(command).\PYGZdl{}
                     send\PYGZus{}cmd=*echo This is the 1st test command.*
\end{sphinxVerbatim}
\begin{itemize}
\item {} 
\sphinxAtStartPar
\$\{result\}{[}0{]} will be \sphinxstylestrong{“This is the 1st test command.”} which is the matched string.

\item {} 
\sphinxAtStartPar
\$\{result\}{[}1{]} will be \sphinxstylestrong{“1st”} which is the first captured string.

\item {} 
\sphinxAtStartPar
\$\{result\}{[}2{]} will be \sphinxstylestrong{“command”} which is the second captured string.

\end{itemize}
\end{quote}
\end{quote}


\chapter{Example}
\label{\detokenize{index:example}}
\begin{sphinxVerbatim}[commandchars=\\\{\}]
*** Settings ***
Documentation    Suite description
Library     QConnectionLibrary.ConnectionManager

*** Test Cases ***
Test SSH Connection
    \PYGZsh{} Create config for connection.
    \PYGZdl{}\PYGZob{}config\PYGZus{}string\PYGZcb{}=    catenate
    ...  \PYGZob{}
    ...   \PYGZdq{}address\PYGZdq{}: \PYGZdq{}127.0.0.1\PYGZdq{},
    ...   \PYGZdq{}port\PYGZdq{}: 8022,
    ...   \PYGZdq{}username\PYGZdq{}: \PYGZdq{}root\PYGZdq{},
    ...   \PYGZdq{}password\PYGZdq{}: \PYGZdq{}\PYGZdq{},
    ...   \PYGZdq{}authentication\PYGZdq{}: \PYGZdq{}password\PYGZdq{},
    ...   \PYGZdq{}key\PYGZus{}filename\PYGZdq{}: null
    ...  \PYGZcb{}
    log to console       \PYGZbs{}nConnecting with configurations:\PYGZbs{}n\PYGZdl{}\PYGZob{}config\PYGZus{}string\PYGZcb{}
    \PYGZdl{}\PYGZob{}config\PYGZcb{}=             evaluate        json.loads(\PYGZsq{}\PYGZsq{}\PYGZsq{}\PYGZdl{}\PYGZob{}config\PYGZus{}string\PYGZcb{}\PYGZsq{}\PYGZsq{}\PYGZsq{})    json

    \PYGZsh{} Connect to the target with above configurations.
    connect             conn\PYGZus{}name=test\PYGZus{}ssh
    ...                 conn\PYGZus{}type=SSHClient
    ...                 conn\PYGZus{}conf=\PYGZdl{}\PYGZob{}config\PYGZcb{}

    \PYGZsh{} Send command \PYGZsq{}cd ..\PYGZsq{} and \PYGZsq{}ls\PYGZsq{} then wait for the response \PYGZsq{}.*\PYGZsq{} pattern.
    send command                conn\PYGZus{}name=test\PYGZus{}ssh
    ...                         command=cd ..

    \PYGZdl{}\PYGZob{}res\PYGZcb{}=     verify                  conn\PYGZus{}name=test\PYGZus{}ssh
    ...                                 search\PYGZus{}pattern=.*
    ...                                 send\PYGZus{}cmd=ls
    log to console     \PYGZdl{}\PYGZob{}res\PYGZcb{}

    \PYGZsh{} Disconnect
    disconnect  test\PYGZus{}ssh
\end{sphinxVerbatim}


\chapter{Contribution Guidelines}
\label{\detokenize{index:contribution-guidelines}}
\sphinxAtStartPar
QConnectBaseLibrary is designed for ease of making an extension library. By that way you can take advantage of the QConnectBaseLibrary’s
infrastructure for handling your own connection protocal. For creating an extension library for QConnectBaseLibrary, please following below
steps.
\begin{enumerate}
\sphinxsetlistlabels{\arabic}{enumi}{enumii}{}{.}%
\item {} 
\sphinxAtStartPar
Create a library package which have the prefix name is \sphinxstylestrong{robotframework\sphinxhyphen{}qconnect\sphinxhyphen{}}\sphinxstyleemphasis{{[}your specific name{]}}.

\item {} 
\sphinxAtStartPar
Your hadling connection class should be derived from \sphinxstylestrong{QConnectionLibrary.connection\_base.ConnectionBase}  class.

\item {} 
\sphinxAtStartPar
In your \sphinxstyleemphasis{Connection Class}, override below attributes and methods:

\end{enumerate}
\begin{quote}
\begin{itemize}
\item {} 
\sphinxAtStartPar
\sphinxstylestrong{\_CONNECTION\_TYPE}: name of your connection type. It will be the input of the conn\_type argument when using \sphinxstylestrong{connect} keyword. Depend on the type name, the library will detemine the correct connection handling class.

\item {} 
\sphinxAtStartPar
\sphinxstylestrong{\_\_init\_\_(self, \_mode, config)}: in this constructor method, you should:

\end{itemize}
\begin{itemize}
\item {} 
\sphinxAtStartPar
Prepare resource for your connection.

\item {} 
\sphinxAtStartPar
Initialize receiver thread by calling \sphinxstylestrong{self.\_init\_thread\_receiver(cls.\_socket\_instance, mode=””)} method.

\item {} 
\sphinxAtStartPar
Configure and initialize the lowlevel receiver thread (if it’s necessary) as below

\begin{sphinxVerbatim}[commandchars=\\\{\}]
\PYG{n+nb+bp}{self}\PYG{o}{.}\PYG{n}{\PYGZus{}llrecv\PYGZus{}thrd\PYGZus{}obj} \PYG{o}{=} \PYG{k+kc}{None}
 \PYG{n+nb+bp}{self}\PYG{o}{.}\PYG{n}{\PYGZus{}llrecv\PYGZus{}thrd\PYGZus{}term} \PYG{o}{=} \PYG{n}{threading}\PYG{o}{.}\PYG{n}{Event}\PYG{p}{(}\PYG{p}{)}
 \PYG{n+nb+bp}{self}\PYG{o}{.}\PYG{n}{\PYGZus{}init\PYGZus{}thrd\PYGZus{}llrecv}\PYG{p}{(}\PYG{n+nb+bp}{cls}\PYG{o}{.}\PYG{n}{\PYGZus{}socket\PYGZus{}instance}\PYG{p}{)}
\end{sphinxVerbatim}

\item {} 
\sphinxAtStartPar
Incase you use the lowlevel receiver thread. You should implement the \sphinxstylestrong{thrd\_llrecv\_from\_connection\_interface()} method. This method is a mediate layer which will receive the data from connection at the very beginning, do some process then put them in a queue for the \sphinxstylestrong{receiver thread} above getting later.

\item {} 
\sphinxAtStartPar
Create the queue for this connection (use Queue.Queue).

\end{itemize}
\begin{itemize}
\item {} 
\sphinxAtStartPar
\sphinxstylestrong{connect()}: implement the way you use to make your own connection protocol.

\item {} 
\sphinxAtStartPar
\sphinxstylestrong{\_read()}: implement the way to receive data from connection.

\item {} 
\sphinxAtStartPar
\sphinxstylestrong{\_write()}: implement the way to send data via connection.

\item {} 
\sphinxAtStartPar
\sphinxstylestrong{disconnect()}: implement the way you use to disconnect your own connection protocol.

\item {} 
\sphinxAtStartPar
\sphinxstylestrong{quit()}: implement the way you use to quit connection and clean resource.

\end{itemize}
\end{quote}


\chapter{Configure Git and correct EOL handling}
\label{\detokenize{index:configure-git-and-correct-eol-handling}}
\sphinxAtStartPar
Here you can find the references for \sphinxhref{https://help.github.com/articles/dealing-with-line-endings/}{Dealing with line
endings}.

\sphinxAtStartPar
Every time you press return on your keyboard you’re actually inserting
an invisible character called a line ending. Historically, different
operating systems have handled line endings differently. When you view
changes in a file, Git handles line endings in its own way. Since you’re
collaborating on projects with Git and GitHub, Git might produce
unexpected results if, for example, you’re working on a Windows machine,
and your collaborator has made a change in OS X.

\sphinxAtStartPar
To avoid problems in your diffs, you can configure Git to properly
handle line endings. If you are storing the .gitattributes file directly
inside of your repository, than you can asure that all EOL are manged by
git correctly as defined.


\chapter{Sourcecode Documentation}
\label{\detokenize{index:sourcecode-documentation}}
\sphinxAtStartPar
For investigating sourcecode, please refer to \sphinxhref{docs/html/index.html}{QConnectBaseLibrary
Documentation}


\chapter{Feedback}
\label{\detokenize{index:feedback}}
\sphinxAtStartPar
If you have any problem when using the library or think there is a
better solution for any part of the library, I’d love to know it, as
this will all help me to improve the library. Please don’t hesitate
to contact me..

\sphinxAtStartPar
Do share your valuable opinion, I appreciate your honest feedback!


\chapter{About}
\label{\detokenize{index:about}}

\section{Maintainers}
\label{\detokenize{index:maintainers}}
\sphinxAtStartPar
{\color{red}\bfseries{}\textasciigrave{}Nguyen Huynh Tri Cuong\textasciigrave{}\_\_}


\section{Contributors}
\label{\detokenize{index:contributors}}
\sphinxAtStartPar
{\color{red}\bfseries{}\textasciigrave{}Nguyen Huynh Tri Cuong\textasciigrave{}\_\_}

\sphinxAtStartPar
{\color{red}\bfseries{}\textasciigrave{}Thomas Pollerspoeck\textasciigrave{}\_\_}


\chapter{License}
\label{\detokenize{index:license}}
\sphinxAtStartPar
robotframework\sphinxhyphen{}qconnect\sphinxhyphen{}base is open source software provided under the {\color{red}\bfseries{}\textasciigrave{}Apache License
2.0\textasciigrave{}\_\_}.


\chapter{QConnect base library’s API!}
\label{\detokenize{index:qconnect-base-library-s-api}}

\section{QConnectionLibrary package}
\label{\detokenize{QConnectionLibrary:qconnectionlibrary-package}}\label{\detokenize{QConnectionLibrary::doc}}

\subsection{Module contents}
\label{\detokenize{QConnectionLibrary:module-contents}}

\chapter{Indices and tables}
\label{\detokenize{index:indices-and-tables}}\begin{itemize}
\item {} 
\sphinxAtStartPar
\DUrole{xref,std,std-ref}{genindex}

\item {} 
\sphinxAtStartPar
\DUrole{xref,std,std-ref}{modindex}

\item {} 
\sphinxAtStartPar
\DUrole{xref,std,std-ref}{search}

\end{itemize}



\renewcommand{\indexname}{Index}
\printindex
\end{document}